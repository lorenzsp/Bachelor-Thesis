
In the linearized approximation, where gravitational fields are weak and velocities are nonrelativistic, we showed that it is straightforward to derive a relationship between the matter dynamics and the emission of gravitational waves, thus obtaining the quadrupole formula.
However, the strongest gravitational-wave signals come from highly compact systems that evolve at relativistic speeds, where the linearized assumptions do not apply. 
Therefore, gravitational-wave detectors find more likely an event which has a powreful signals.
Thus, it is important to be able to calculate gravitational-wave emission accurately for processes such as black hole or neutron star inspiral and merger.
Such problems cannot be solved analytically and instead are modeled by numerical relativity to compute the gravitational field near the source.\\
In this section we study the gravitational-wave signals obtained from numerical simulations of compact binaries, using the Einstein Toolkit, an open-source computational infrastructure for numerical relativity based on Cactus Framework.
Although we do not give a detailed explanation of the numerical methods used to solve the proposed simulations, we study the gravitational-wave signals and give an overview of different physical quantities which describe the binary sources.
\subsection{Binary Black holes}
\subsection{Binary Neutron stars}