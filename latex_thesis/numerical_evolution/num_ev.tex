
In the linearized approximation, where gravitational fields are weak and velocities are nonrelativistic, we showed that it is straightforward to derive a relationship between the matter dynamics and the emission of gravitational waves, thus obtaining the quadrupole formula.
However, the strongest gravitational-wave signals come from highly compact systems that evolve at relativistic speeds, where the linearized assumptions do not apply. 
Therefore, gravitational-wave detectors find more likely an event which has a powreful signals.
Thus, it is important to be able to calculate gravitational-wave emission accurately for processes such as black hole or neutron star inspiral and merger.
Such problems cannot be solved analytically and instead are modeled by numerical relativity to compute the gravitational field near the source.\\
In this section we study the gravitational-wave signals obtained from numerical simulations of compact binaries, using the Einstein Toolkit, an open-source computational infrastructure for numerical relativity based on Cactus Framework.\\
The Cactus framework [11–13] is a general framework for the development of portable, modular applications, wherein programs are split into components (called thorns) with clearly defined dependencies and interactions. 
Thorns are typically developed independently and do not directly interact with each other. 
Cactus simulations require an ex
\\
A thorough description of the numerical methods used to perform the simulations can be found here, we depict the initial conditions of our simulations and we briefly mention the used thorns.
Rather than analyzing the algorithms of th Einstein Toolkit, the purpose of the following sections is to study the gravitational-wave signals of binaries black holes (BBH) and neutron stars (BNS). 

\subsection{Binary Black holes}
We simulate the evolution of two equal-mass black holes with quasi-equilirbium initial conditions.
We use the 




\subsection{Binary Neutron stars}